% Preamble
\documentclass[11pt]{PyRollDocs}
\usepackage{textcomp}
\usepackage{stmaryrd}


\addbibresource{refs.bib}

% Document
\begin{document}

    \title{The Sander Spreading PyRoll Plugin}
    \author{Christoph Renzing}
    \date{\today}

    \maketitle

    This plugin provides a spreading modelling approach with Sander's formula for flat rolling.


    \section{Model approach}\label{sec:model-approach}

    \subsection{Sanders's spread equation}\label{subsec:sander's-spread-equation}

    \textcite{Sander1976, Sander1978} proposed \autoref{eq:sander} for estimation of spreading in flat rolling,
    where $\gamma = \frac{h_1}{h_0}$ is the compression. $h$ and $b$ are height and width of the workpiece with the indices
    0 and 1 denoting the incoming respectively the outgoing profile. $a$, $c$, $d$ and $f$ are correction
    coefficients for temperature, velocity, material and friction, respectively.

    \begin{equation}
        \beta = \frac{b_1}{b_0} = a \times c \times d \times f \times \gamma^{-w}
        \label{eq:sander}
    \end{equation}

    $w$ is the spread exponent, by \textcite{Sander1976} is given in \autoref{eq:exponent}, where $R$ is the roll radius.

    \begin{equation}
        w = 10^{ -0.76 \left( \frac{h_0}{b_0} \right)^{0.39} \left(\frac{b_0}{\sqrt{R \Delta h}} \right)^{0.12} \left( \frac{b_0}{R} \right)^{0.59} }
        \label{eq:exponent}
    \end{equation}

    \noindent The temperature coefficient $a$ is implement using the below condition for various temperature ranges.

    \begin{equation}
        a =
        \begin{cases}
            1.005          & \text{if }  \qty{700}{\celsius} \leq \vartheta \leq \qty{950}{\celsius}\\
            1              & \text{otherwise}
        \end{cases}
        \label{eq:temperature-coefficient}
    \end{equation}

    \noindent The velocity coefficient $c$ can be assumed as below in dependence on the velocity $v$.

    \begin{equation}
        c = 1 - 0.0033 v \left( 1 - \frac{1}{\beta} \right)
        \label{eq:velocity-coefficient}
    \end{equation}

    The origin of this equation was first given by \textcite{Hill1955} which derived the equation from plastic stress-strain equations from \textcite{Mises1913}.


    \section{Usage instructions}\label{sec:usage-instructions}

    The plugin can be loaded under the name \texttt{pyroll\_sander\_spreading}.

    An implementation of the \lstinline{width} hook on \lstinline{RollPass.out_profile} is provided,
    calculating the width using the equivalent rectangle approach and Sander's model.

    Several additional hooks on \lstinline{RollPass} are defined, which are used in spread calculation, as listed in \autoref{tab:hookspecs}.
    Base implementations of them are provided, so it should work out of the box.
    For \lstinline{sander_exponent}, \lstinline{sander_temperature_coefficient} and \lstinline{sander_velocity_coefficient}
    the equations~\ref{eq:exponent},~\ref{eq:temperature-coefficient} and~\ref{eq:velocity-coefficient} are implemented.
    The others default to \num{1}.
    Provide your own hook implementations or set attributes on the \lstinline{RollPass} instances to alter the spreading behavior.

    \begin{table}
        \centering
        \caption{Hooks specified by this plugin. Symbols as in \autoref{eq:sander}.}
        \label{tab:hookspecs}
        \begin{tabular}{ll}
            \toprule
            Hook name                                 & Meaning                                \\
            \midrule
            \texttt{sander\_temperature\_coefficient} & temperature correction coefficient $a$ \\
            \texttt{sander\_velocity\_coefficient}    & velocity correction coefficient $c$    \\
            \texttt{sander\_material\_coefficient}    & material correction coefficient $d$    \\
            \texttt{sander\_friction\_coefficient}    & friction correction coefficient $f$    \\
            \texttt{sander\_exponent}                 & spread exponent $w$                    \\
            \bottomrule
        \end{tabular}
    \end{table}

    \printbibliography

\end{document}