% Preamble
\documentclass[11pt]{PyRollDocs}
\usepackage{textcomp}
\usepackage{stmaryrd}

\addbibresource{refs.bib}

% Document
\begin{document}

    \title{The Sander Spreading PyRoll Plugin}
    \author{Christoph Renzing}
    \date{\today}

    \maketitle

    This plugin provides a spreading modelling approach with Sander's formula for flat rolling.


    \section{Model approach}\label{sec:model-approach}

    \subsection{Sanders's spread equation}\label{subsec:sander's-spread-equation}

    \textcite{Sander1976, Sander1978} proposed \autoref{eq:sander} for estimation of spreading in flat rolling.
    $h$ and $b$ are height and width of the workpiece with the indices 0 and 1 denoting the incoming respectively the outgoing profile.

    \begin{equation}
        \beta = \frac{b_1}{b_0} = \frac{h_0}{h_1} ^{w}
        \label{eq:sander}
    \end{equation}
    
    $w$ is the spread exponent, by \textcite{Sander1976} is given in \autoref{eq:exponent}, where $R$ is the roll radius.

    \begin{equation}
        w = 10^{ -0.76 \left( \frac{h_0}{b_0} \right)^{0.39} \left(\frac{b_0}{\sqrt{R \Delta h}} \right)^{0.12} \left( \frac{b_0}{R} \right)^{0.59} }
        \label{eq:exponent}
    \end{equation}


    \section{Usage instructions}\label{sec:usage-instructions}

    The plugin can be loaded under the name \texttt{pyroll\_sander\_spreading}.

    An implementation of the \lstinline{spread} hook on \lstinline{RollPass} is provided, calculating the spread using Sander's model.
    For \lstinline{sander_exponent} the equations~\ref{eq:exponent} is implemented.
    Provide your own hook implementations or set attributes on the \lstinline{RollPass} instances to alter the spreading behavior.

    \printbibliography

\end{document}